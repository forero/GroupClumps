\documentclass{article}
\usepackage{natbib}
\newcommand{\apj}{ApJ}  
\newcommand{\apjs}{ApJS}  
\newcommand{\apjl}{ApJL}  
\newcommand{\aj}{AJ}  
\newcommand{\mnras}{MNRAS}  
\newcommand{\mnrassub}{MNRAS accepted}  
\newcommand{\aap}{A\&A}  
\newcommand{\aaps}{A\&AS}  
\newcommand{\araa}{ARA\&A}  
\newcommand{\nat}{Nature}  
\newcommand{\physrep}{PhR}
\newcommand{\pasp}{PASP}    
\newcommand{\pasj}{PASJ}    
\newcommand{\hMsun}{{\ifmmode{h^{-1}{\rm {M_{\odot}}}}\else{$h^{-1}{\rm{M_{\odot}}}$}\fi}}  
\newcommand{\Msun}{{\ifmmode{{\rm {M_{\odot}}}}\else{${\rm{M_{\odot}}}$}\fi}} 
\newcommand{\hMpc}{{\ifmmode{h^{-1}{\rm Mpc}}\else{$h^{-1}$Mpc }\fi}}  
\newcommand{\hkpc}{{\ifmmode{h^{-1}{\rm kpc}}\else{$h^{-1}$kpc }\fi}}  
\newcommand{\kms}{\,km~s$^{-1}$}
\title{Simulation Description}
\author{Jaime E. Forero-Romero\\ Departamento de F\'{i}sica, Universidad de los Andes\\ Cra. 1 No. 18A-10, Edificio Ip, Bogot\'a, Colombia}



\begin{document}
\maketitle
\section{Simulation Description}
We have used the data from a large N-body cosmological simulation
dubbed \verb"Multidark". The data is available to the public through a
database interface first presented by \cite{Riebe11}. 

Here we summarize the main characteristics in the simulations. More
details can be found in \cite{Prada12}. The simulation was run using
an Adaptive-Mesh Refinement (AMR) code called ART. Details about the
technical aspects and comparisons with other N-body codes are given in
\cite{Klypin09}.   

The simulations follows the non-linear evolution of a dark matter
density field sampled with $2048^3$ particles in a volume of
$1000$\hMpc. The physical resolution of the simulation is almost
constant in time $\sim 7$\hkpc between redshifts $z=0-8$. The
cosmological parameters in the simulation are $\Omega_m=0.27$,
$\Omega_{\Lambda}=0.73$, $n_{s}=0.95$, $h=0.70$ and $\sigma_8=0.82$ for
the matter density, dark energy density, slope of the matter
fluctuations, the Hubble constant at $z=0$ in units of 100km s$^{-1}$
Mpc$^{-1}$ and the normalization of the power spectrum, respectively. 
With these characteristics the mass per simulation particle is
$m_p=8.63\times 10^{9}$\hMsun, which means that group-like halos of masses
$\sim 10^{13}$\hMsun are sampled with at least 1100 particles. 

Dark matter halos are identifyed using a Bound-Density-Maxima
algorithm (BDM). The code finds density maxima in spherical
regions. For each maxima the code estimates the radius that encloses
an specified value for the density

The halos we have used take define the spherical regions that enclose
within a radius $R_{200}$ an average density $\Delta=200$ times larger than the
critical density of the Universe, $\rho_{\rm cr}$. This allows the
code to relate the radius $R_{200}$ and the enclosed mass $M_{200}$ as
follows: 

\begin{equation}
M_{200} = \frac{4\pi}{3}\Delta \rho_{\rm cr}(z)R^{3}
\end{equation}

\subsection{Concentration Estimates}

The estimation for the concentration values is done using an
analytical property of the NFW profile that relates the circcular
velocity at the virial radius:
\begin{equation}
V_{200} = \left(\frac{GM_{200}}{R_{200}}\right)^{1/2},
\end{equation}
with the maximum circular velocity 
\begin{equation}
V_{\rm max}^{2} = {\rm max}\left[\frac{GM(<r)}{r}\right].
\end{equation}


The $V_{\rm max}/V_{200}$ velocity ratio is used to determine the halo
concentration, $c$, by using the following relation \citep{Bolshoi}:

\begin{equation}
\frac{V_{\rm max}}{V_{200}} = \left(\frac{0.216 c}{c}\right)^{1/2}
\end{equation}

where $f(c)$ is
\begin{equation}
f(x) = \ln(1+c) - \frac{c}{(1+c)}.
\end{equation}

For each BDM overdensity the $V_{\rm max}/V_{200}$ ratio in order to
find the concentration $c$ by solving numerically the previous two
equations. 

This method provides a much robust estimate of the concentration
compared to a radial fitting to the NFW profile which is strongly
dependent on the radial range used for the fit
\citep{Bolshoi,Meneghetti13}. Comparison of these two methods to
using halos where the NFW functional fit is robust yield a small
systematic offset of $(5-15)\%$, with the concentration  derived by
the velocity ratio method being higher \citep{Prada12}. 

\subsection{Halo Sample Selection}

We have used the observational data to define $10$ redshift bins
as defined in Table 1.

We query the MultiDark database to obtain all the information for
halos with rms velocities in the range $300$\kms$<V_{\rm
  rms}<1000$\kms. We use the values for $V_{\rm rms}$ as a proxy for
the velocity dispersion inferred in observed lenses. For each redshift
we construct a relationship $V_{\rm rms}-c$ by binning the halos in
the catalog in bins of $50$\kms width. For each velocity bin the
average and standar deviation of the concentration is calculated. 

Additionally we construct a new halo catalog using halos from all the
mock catalogs as to match the shape of the observational redshift
distribution for the lenses. We proceed as follows, for each redshift
bin we count the number of observed lenses and multiply that number by
$10^5$, then we randomly select as many halos from the catalogs. This
allows us to have a simulated sample $10^5$ times larger as the
observational one, with the same redshift distribution. From this new
catalog we construct a new $V_{\rm rms}-c$ relationship in the same
way as described in the last paragraph.

\begin{table}
\begin{tabular}{ccc}
center & Simulation & Total Number \\
redshift bin & Snapshot & Number of Halos\\
0.05 & 82 & 781764\\
0.16 & 76 & 816422\\
0.28 & 70 & 854495\\
0.40 & 66 & 869768\\
0.52 & 62 & 901588\\
0.64 & 60 & 909205\\
0.75 & 56 & 920882\\
0.87 & 54 & 923512\\
0.99 & 52 & 904938\\
1.10 & 50 & 898774\\
\end{tabular}
\caption{First column, center redshift bins used to query the MultiDark data
  base. Second column, corresponding snapshot number in the
  simulation. Third column, number of halos in the whole volume box
  with RMS velocities in the range $300$\kms-$1000$\kms.}
\end{table}



\bibliographystyle{abbrvnat}
\bibliography{references}

\end{document}
