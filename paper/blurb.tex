\documentclass{article}
\usepackage{natbib}
\newcommand{\apj}{ApJ}  
\newcommand{\apjs}{ApJS}  
\newcommand{\apjl}{ApJL}  
\newcommand{\aj}{AJ}  
\newcommand{\mnras}{MNRAS}  
\newcommand{\mnrassub}{MNRAS accepted}  
\newcommand{\aap}{A\&A}  
\newcommand{\aaps}{A\&AS}  
\newcommand{\araa}{ARA\&A}  
\newcommand{\nat}{Nature}  
\newcommand{\physrep}{PhR}
\newcommand{\pasp}{PASP}    
\newcommand{\pasj}{PASJ}    
\newcommand{\hMsun}{{\ifmmode{h^{-1}{\rm {M_{\odot}}}}\else{$h^{-1}{\rm{M_{\odot}}}$}\fi}}  
\newcommand{\Msun}{{\ifmmode{{\rm {M_{\odot}}}}\else{${\rm{M_{\odot}}}$}\fi}} 
\newcommand{\hMpc}{{\ifmmode{h^{-1}{\rm Mpc}}\else{$h^{-1}$Mpc }\fi}}  
\newcommand{\hkpc}{{\ifmmode{h^{-1}{\rm kpc}}\else{$h^{-1}$kpc }\fi}}  
\newcommand{\kms}{\,km~s$^{-1}$}
\title{Simulation Description}
\author{Jaime E. Forero-Romero\\ Departamento de F\'{i}sica, Universidad de los Andes\\ Cra. 1 No. 18A-10, Edificio Ip, Bogot\'a, Colombia}



\begin{document}
\maketitle




   
\section{Simulations}\label{Simulations}

We used a large N-body simulation dubbed \verb"Multidark"
to extract statistics of halo parameters on cosmological scales. The
data we use for this paper is publicly available through a database
interface first presented by \cite{Riebe11}. Here we summarize the
main characteristics of the \verb"Multidark" volume. More details can
be found in \cite{Prada12}.  

The simulation was run using  an Adaptive-Mesh Refinement (AMR) code
called ART. Details about the technical aspects and comparisons with
other N-body codes are given in \cite{Klypin09}.   The simulations
follows the non-linear evolution of a dark matter density field
sampled with $2048^3$ particles in a volume of $1000$\hMpc. The
physical resolution of the simulation is almost constant in time $\sim
7$\hkpc between redshifts $z=0-8$. The cosmological parameters in the
simulation are $\Omega_m=0.27$, $\Omega_{\Lambda}=0.73$, $n_{s}=0.95$,
$h=0.70$ and $\sigma_8=0.82$ for the matter density, dark energy
density, slope of the matter fluctuations, the Hubble constant at
$z=0$ in units of 100km s$^{-1}$ Mpc$^{-1}$ and the normalization of
the power spectrum, respectively.  These cosmological parameters are
consistent with the results from WMAP5 and WMAP7
\citep{Komatsu2009,Jarosik2011}. With these characteristics the mass
per simulation particle is $m_p=8.63\times  10^{9}$\hMsun, which means
that group-like halos of masses $\sim 10^{13}$\hMsun are sampled with
at least 1100 particles.   

Dark matter halos are identified using a Bound-Density-Maxima
algorithm (BDM). The code starts by finding the density maxima at the
particles' positions in the simulation volume. For each maxima
it finds the radius $R_{200}$ of a sphere containing a mass
overdensity given by 

\begin{equation}\label{eq:M200}
M_{200} = \frac{4\pi}{3}\Delta \rho_{\rm cr}(z)R_{200}^{3}
\end{equation}

where $\rho_{\rm cr}$ is the critical density of the Universe and
$\Delta=200$ is the desired overdensity. This procedure allows for the
detection of both halos and sub-halos. In our analysis we kept only
the halos. 

\subsection{Concentration Estimates}

The estimation for the concentration values is done using an
analytical property of the NFW profile (see Sect.\,\ref{SL_Ana}) that relates the circular
velocity at the virial radius:
\begin{equation}
V_{200} = \left(\frac{GM_{200}}{R_{200}}\right)^{1/2},
\end{equation}
with the maximum circular velocity 
\begin{equation}
V_{\rm max}^{2} = {\rm max}\left[\frac{GM(<r)}{r}\right].
\end{equation}


The $V_{\rm max}/V_{200}$ velocity ratio is used to determine the halo
concentration, $c$ (the ratio between $R_{200}$ and the scale radii of
the NFW profile), using the following relation \citep{Bolshoi}:

\begin{equation}
\frac{V_{\rm max}}{V_{200}} = \left(\frac{0.216 c}{c}\right)^{1/2}
\end{equation}

where $f(c)$ is
\begin{equation}
f(x) = \ln(1+c) - \frac{c}{(1+c)}.
\end{equation}

For each BDM overdensity the $V_{\rm max}/V_{200}$ ratio in order to
find the concentration $c$ by solving numerically the previous two
equations.  This method provides a robust estimate of the concentration
compared to a radial fitting to the NFW profile, which is strongly
dependent on the radial range used for the fit
\citep{Bolshoi,Meneghetti13}. Comparison of these two methods to
using halos where the NFW functional fit yield a small systematic
offset of $(5-15)\%$, with the concentration  derived by the velocity
ratio method being higher \citep{Prada12}.  

\bibliographystyle{abbrvnat}
\bibliography{references}

\end{document}
