\documentclass{article}
\usepackage{natbib}
\newcommand{\apj}{ApJ}  
\newcommand{\apjs}{ApJS}  
\newcommand{\apjl}{ApJL}  
\newcommand{\aj}{AJ}  
\newcommand{\mnras}{MNRAS}  
\newcommand{\mnrassub}{MNRAS accepted}  
\newcommand{\aap}{A\&A}  
\newcommand{\aaps}{A\&AS}  
\newcommand{\araa}{ARA\&A}  
\newcommand{\nat}{Nature}  
\newcommand{\physrep}{PhR}
\newcommand{\pasp}{PASP}    
\newcommand{\pasj}{PASJ}    
\newcommand{\hMpc}{{\ifmmode{h^{-1}{\rm Mpc}}\else{$h^{-1}$Mpc }\fi}}  
\newcommand{\hMpc}{{\ifmmode{h^{-1}{\rm kpc}}\else{$h^{-1}$kpc }\fi}}  
\title{Simulation Description}
\author{Jaime E. Forero-Romero\\ Departamento de F\'{i}sica, Universidad de los Andes\\ Cra. 1 No. 18A-10, Edificio Ip, Bogot\'a, Colombia}



\begin{document}
\maketitle
We have used the data from a large N-body cosmological simulation
dubbed \verb"Multidark". The data is available to the public through a
database interface first presented by \cite{Riebe11}. 

Here we summarize the main characteristics in the simulations. More
details can be found in \cite{Prada12}. The simulation was run using
an Adaptive-Mesh Refinement (AMR) code called ART. Details about the
technical aspects and comparisons with other N-body codes are given in
\cite{Klypin09}.   

The simulations follows the non-linear evolution of a dark matter
density field sampled with $2048^3$ particles in a volume of
$1000$\hMpc. The physical resolution of the simulation is almost
constant in time $\sim 7$\hkpc between redshifts $z=0-8$. The
cosmological parameters in the simulation are $\Omega_m=0.27$,
$\Omega_{\Lambda}=0.73$, $n_{s}=0.95$, $h=0.70$ and $\sigma_8=0.82$ for
the matter density, dark energy density, slope of the matter
fluctuations, the Hubble constant at $z=0$ in units of 100km s$^{-1}$
Mpc$^{-1}$ and the normalization of the power spectrum, respectively. 
With these characteristics the mass per simulation particle is
$m_p=8.63\times 10^{9}$\hMsun, which means that group-like halos of masses
$\sim 10^{13}$\hMsun are sampled with at least 1100 particles. 

Dark matter halos are identifyed using a Bound-Density-Maxima
algorithm (BDM). The code finds density maxima in spherical
regions. For each maxima the code estimates the radius that encloses
an specified value for the density

The halos we have used take define the spherical regions that enclose
within a radius $R_{200}$ an average density $\Delta=200$ times larger than the
critical density of the Universe, $\rho_{\rm cr}$. This allows the
code to relate the radius $R_{200}$ and the enclosed mass $M_{200}$ as
follows: 

\begin{equation}
M_{200} = \frac{4\pi}{3}\Delta \rho_{\rm cr}(z)R^{3}
\end{equation}

\bibliographystyle{abbrvnat}
\bibliography{references}

\end{document}
